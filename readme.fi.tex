\documentclass[a4paper,12pt]{article}
\date{23.02.2008}
\usepackage{pslatex}
\usepackage[utf8]{inputenc}
\usepackage[finnish]{babel}
\usepackage[T1]{fontenc}
\begin{document}

\section{Suomalainen laskupohja \LaTeX-formaatissa}

\subsection{Tarkoitus}

Olen tehnyt tämän laskupohjan laskutussovellustani varten. \LaTeX{} on erittäin 
kätevä vaihtoehto GUI-pohjaisille sovelluksille erityisesti silloin, jos on 
tarve tehdä laskuja koneellisesti. Laskutussovellukseni on jo melkein valmis. 
Tämä mallipohja oli vaivalloisin vaihe siinä projektissa. Valmistuessaan 
sovelluksen on tarkoitus lähettää PDF-muodossa laskuja automaattisesti 
asiakastietojen mukaan. Käyttäjän tarvitsee lähinnä vain ylläpitää rekisteriä 
ja postittaa tulostettuja laskuja.

\subsection{Toteutus}

Kyseessä on Finanssialan Keskusliiton määrityksien mukaan \footnote{Uusi 
tilisiirto-opas (http://www.fkl.fi/)} toteutetulla tilisiirtolomakkeella 
varustettu laskupohja. Tämä muoto tilisiirtolomakkeesta on tarkoitettu 
SEPA-siirtymäkauden lomakkeeksi 2008-1010.

\subsection{Muokkausohjeet}

Mallipohjaa saa kuka tahansa käyttää vapaasti zlib-lisenssin ehtojen 
mukaisesti. Huomaa kuitenkin viivakoodin GPL-lisenssi!

Käyttäjä saa muokata varsinaisen laskuosuuden mielensä mukaan, mutta 
tilisiirtolomakkeen osuudella on tarkat määritykset. Nykyisen mallin pitäisi 
olla jo täysin määritysten mukainen, lukuunottamatta käytettyjä malliarvoja, 
jotka varsinkin viivakoodin osalta ovat vielä virheelliset. Lisäksi mallista 
puuttuu vielä määrittelyssä mainittu vapaaehtoinen tuotekoodi. Jos huomaat 
laskupohjassa virheitä, kuulen niistä mielelläni.

Laskun otsikko-osuus on suunniteltu sellaiseksi, että laskun voi sellaisenaan 
postittaa ikkunallisessa kirjekuoressa, jolloin säästytään osoitetietojen 
kirjoittamiselta kirjeeseen erikseen.

\subsection{TEX-tiedoston käsittely}

Tex-tiedosto muutetaan UNIX-järjestelmässä PDF-tiedostoksi esimerkiksi komennolla:
\begin{verbatim}
pdflatex invoice.tex
\end{verbatim}

Käytin itse Kile-sovellusta \footnote{Kile -- an integrated LaTeX environment, 
http://kile.sourceforge.net/} mallipohjan tekemiseen. Kilessä on sisäänrakennettu 
monipuolinen \LaTeX-tiedostojen hallinta ja tekstieditori. Se ei täysin wysiwyg 
\footnote{What You See Is What You Get} ole, mutta huomattavasti helpompi kuin puhdas tekstieditori.

\subsection{Viivakoodi}

Mallipohja käyttää erillistä code128.tex-laajennusta \footnote{Barcode macros 
for the Code 128 standard, 
http://www.ctan.org/tex-archive/help/Catalogue/entries/code128.html} 
viivakoodin toteuttamiseen. Laajennus ei ole minun tekemä. Viivakoodi tulisi 
tehdä pankkiviivakoodistandardin mukaan \footnote{Pankkiviivakoodistandardi 
versio 5.0/15.6.2002, http://www.fkl.fi/}. Se pitäisi myös testauttaa ennen käyttöönottoa.

\subsection{Kehitystehtävät ja ideat}

\begin{itemize}
 \item Malliarvot oikeanlaisiksi
 \item Vapaaehtoinen tuotekoodikenttä
 \item Uudet kielivaihtoehtot suomi-englanti ja ruotsi-englanti
 \item Makrokielillä voisi toteuttaa \LaTeX{} packagen, jolloin mallista tulisi helpommin ylläpidettävä ja laskujen tekemisestä helppoa myös ilman sovellusta.
 \item Makrokielellä arvojen laskeminen
\end{itemize}

\end{document}
